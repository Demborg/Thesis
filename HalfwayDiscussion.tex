% Created 2018-04-06 fre 12:27
% Intended LaTeX compiler: pdflatex
\documentclass[11pt]{article}
\usepackage[utf8]{inputenc}
\usepackage[T1]{fontenc}
\usepackage{graphicx}
\usepackage{grffile}
\usepackage{longtable}
\usepackage{wrapfig}
\usepackage{rotating}
\usepackage[normalem]{ulem}
\usepackage{amsmath}
\usepackage{textcomp}
\usepackage{amssymb}
\usepackage{capt-of}
\usepackage{hyperref}
\author{Axel Demborg}
\date{\today}
\title{Halfway meeting}
\hypersetup{
 pdfauthor={Axel Demborg},
 pdftitle={Halfway meeting},
 pdfkeywords={},
 pdfsubject={},
 pdfcreator={Emacs 25.3.1 (Org mode 9.1.6)}, 
 pdflang={English}}
\begin{document}

\maketitle

\section{High-level Task}
\label{sec:org8aea82c}
Semantic segmentation for Mobile Phones

\section{High-level Approach}
\label{sec:org8d24f1f}
\begin{enumerate}
\item Learn efficient models from scratch
\item Make pretrained models efficient
\end{enumerate}

\section{High-level RQ}
\label{sec:org2bd2f8d}
\begin{enumerate}
\item To what extent is efficiency possible?
\item How does performance transfer from synthesized to real data?
\item Combination of 1) and 2) (Fewer parameters --> small model and good generalizing)
\item Something about video??
\end{enumerate}

\section{Low-level Task:}
\label{sec:org07b3ee4}
Semantic foot segmentation on real/synthesized data

\section{Low-level approach:}
\label{sec:org2880b77}
\begin{itemize}
\item Train on synthesized data
\item Validate with either synthesized data or real data (cross validation?)
\item Test on real or synthesized data to get scores
\end{itemize}

\section{Low-level goals}
\label{sec:org4d7a48d}
\begin{description}
\item[{Transferability}] mean IoU, mean accuracy, DICE?
\item[{Efficiency}] Network size (memory and storage), Speed (FLOPS and Inference time/fps on phone)
\end{description}

\section{Low-level RQ}
\label{sec:org9d817ff}
\begin{itemize}
\item What network architectures are suitable for the task?
\begin{itemize}
\item ENet \url{https://arxiv.org/pdf/1606.02147.pdf}
\item MobileNet with upsampling path \url{https://arxiv.org/pdf/1704.04861.pdf}
\item LinkNet \url{https://arxiv.org/pdf/1707.03718.pdf}
\begin{itemize}
\item LinkNet with depthwice separable convolutions,inspired by MobileNet (smaller, faster)
\end{itemize}
\end{itemize}
\item How should the networks be trained?
\begin{itemize}
\item From scratch
\begin{itemize}
\item Cross entropy loss
\item IoU loss \url{http://www.cs.umanitoba.ca/\~ywang/papers/isvc16.pdf}
\item Some sort of GANs approach??? \url{https://arxiv.org/pdf/1611.08408.pdf}
\end{itemize}
\item With a teacher network (EmilSeg a prebuilt monstrosity)
\begin{itemize}
\item Distillation \url{https://arxiv.org/pdf/1503.02531.pdf}
\item Attention transfer \url{https://arxiv.org/pdf/1612.03928.pdf}
\end{itemize}
\end{itemize}
\item What other things can be done to increase performance?
\begin{itemize}
\item Use temporal aspect of real data
\begin{itemize}
\item Add some momentum to pixels, kind of like persistence of vision in humans \url{https://en.wikipedia.org/wiki/Persistence\_of\_vision}
\item Add LSTM at bottleneck (too slow)
\item Feed last prediction back as additional channels (tested a bit, didn't get it to work)
\end{itemize}
\item Do pretraining on other data
\begin{itemize}
\item Train encoder on say ImageNet to learn visual features?
\item Train for segmentation on something more general (DAVIS?) and finetune for feet?
\end{itemize}
\end{itemize}
\item What else can be done to decrease model size/inference speed? (Don't realy think any of these will pan out since they work poorly together with the hardware/frameworks we have to work with)
\begin{itemize}
\item Quantization
\item HashedNets \url{https://arxiv.org/pdf/1504.04788.pdf}
\end{itemize}
\end{itemize}
\end{document}